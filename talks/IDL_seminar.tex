\documentclass{beamer}       % print frames
%\documentclass[notes=only]{beamer}   % only notes
%\documentclass{beamer}              % only frames

\newcounter{savedenum}
\newcommand*{\saveenum}{\setcounter{savedenum}{\theenumi}}
\newcommand*{\resume}{\setcounter{enumi}{\thesavedenum}}
\usepackage{multienum}
\usepackage[notocbib]{apacite}
	\renewcommand{\APACrefatitle}[2]{#1}
	\renewcommand{\APACrefbtitle}[2]{#1}
	\renewcommand{\APACrefbtitle}[2]{\Bem{#1}}
	\renewcommand{\APACrefaetitle}[2]{[#2]}
	\renewcommand{\APACrefbetitle}[2]{[#2]}
\usepackage{textpos}
\usefonttheme{serif}
%\usepackage{graphicx}
\usetheme{Boadilla}
\usepackage{hyperref}
\usepackage[UKenglish]{babel}
\usepackage{lingmacros}
\setcounter{tocdepth}{1} 
\usecolortheme{orchid}
\title[University of Melbourne]{Exploring the influence of medium and culture on language use in an online support group}
\author[Daniel McDonald]{~\\Daniel McDonald~\\~\\~\\\footnotesize}

\date{IDL, 20/3/2015}

\begin{document}

\addtobeamertemplate{frametitle}{}{%
\begin{textblock*}{100mm}(.775\textwidth,-.5cm)
\includegraphics[scale=.235]{../images/unimelblong}
\end{textblock*}}

\frame{\titlepage}

\begin{frame}
\frametitle{Context}
\begin{itemize}
\item Show you all what my PhD research is about
\item Introduce \emph{Systemic Functional Linguistics} (SFL)
\item Discuss some of the problems \emph{SFL} might help \emph{us} solve
\item Discuss some of the problems \emph{you} might help \emph{me} solve
\end{itemize}
\end{frame}

  \begin{frame}
    \frametitle{Case study: my thesis}
    \begin{itemize}
    \item I'm researching longitudinal linguistic change in a 12 year old bipolar disorder forum
    \item 60,000 posts
    \item 5821 unique usernames
    \item A few veterans with over 1000 posts, most users have 1--5
    \item Rules: no asking for diagnosis, stay `anonymous'
    \item Normative biomedical ideology \cite<c.f.>{vayreda_social_2009}
    \item People come for information and/or social support
    \item Veterans answer questions, welcome new members, speak as `we'
    \end{itemize}
  \end{frame}

 \begin{frame}
    \frametitle{The forum}
    \centering
    \includegraphics[scale=.5]{../images/forum.png}
  \end{frame}

\begin{frame}
 	\frametitle{Thesis methodology: the easy part}
 	
 	\begin{itemize}
 	\item \emph{Wget} to spider the forum
 	\item Extract text from HTML with \emph{Beautiful Soup}
 	\item Spelling normalisation via \emph{PyEnchant} 
 	\item Parse posts for constituency and dependency grammar with \emph{Stanford CoreNLP}
 	\item Create subcorpora based on post count
 	\item Build \texttt{corpkit}, some tools for searching parse trees, manipulating and plotting results: \texttt{pip install corpkit} (Alpha!)
 	\item Develop an \emph{IPython Notebook} interface for working with the corpus\slash toolkit
 	\item \url{https://github.com/interrogator/corpkit}, \url{https://github.com/interrogator/sfl_corpling}
 	\end{itemize}
 \end{frame}


 \begin{frame}
 	\frametitle{Analysing 8m words: the hard part}


 	How do we actually make sense of this huge collection of data?


 \end{frame}
 
 

 \begin{frame}
    \frametitle{    \frametitle{An example: \emph{Katlin09}}}

    Katlin09's early posts look like this:

  \centering \includegraphics[width=0.9\textwidth]{../images/kat1.png}
  \end{frame}

 \begin{frame}
    \frametitle{Katlin09}
    	
    	... and here's one of her later posts...

        \centering \includegraphics[width=0.9\textwidth]{../images/kat2.png}
  \end{frame}


 \begin{frame}
 	\frametitle{What's happening with her language use?}
 	
 	\begin{itemize}
 	\item Has her language use changed over time?
 	\item Does this same change happen to other members?
 	\item What \emph{causes} the change?
 	\item If many things cause the change, how can we distinguish what forces are acting and when?
 	\end{itemize}

 	To answer these kinds of questions, we need a \emph{theory of language}
 \end{frame}
 
\begin{frame}
    \frametitle{\emph{Systemic Functional Linguistics} (SFL)}

\citeA{halliday_introduction_2004} conceptualise language as:
\begin{enumerate}
 \item Systemic: a sign-\emph{system}, from which users select realisations
 \item Functional: structured to achieve meaningful social goals
 \item Constitutive of context: context is mapped onto abstractions of grammatical systems (mood, transitivity and theme)
 \begin{itemize}
 \item \textbf{Context is in text.}
 \end{itemize}
 \end{enumerate} 

\end{frame}

\begin{frame}
    \frametitle{Overview of SFL}
    \centering
    \includegraphics[width=0.90\textwidth]{../images/egginsfixed.jpg}
\end{frame}

\begin{frame}
\frametitle{SFL: Introduction}
SFL argues that we can break communication into \emph{strata}
\begin{itemize}
\item Context:
\begin{itemize}
    	\item Genre
    	\item Register
\end{itemize}
\item Language:
\begin{itemize}    	
    	\item Discourse-semantic
    	\item Lexicogrammatical
    	\item Phonological/typographical
    \end{itemize}
\end{itemize}
\end{frame}
 
\begin{frame}
\frametitle{SFL: \emph{Metafunctions}}
In SFL, language/images are communicating \emph{up to} three different \emph{kinds} of meaning simultaneously:
\begin{itemize}
	\item Interpersonal meaning: roles and relationships between people
	\item Ideational meaning: what happens to whom, under what circumstances?
	\item Textual meaning: the role that language is playing in the interaction
\end{itemize}
    \end{frame}

 \begin{frame}
    \frametitle{Overview of SFL}
    \centering
    \includegraphics[width=0.90\textwidth]{../images/egginsfixed.jpg}
\end{frame}

 \begin{frame}
 \frametitle{Metafunctions in the lexicogrammar}
 Each function is accomplished through different linguistic subsystems at the level of lexicogrammar.
 \begin{itemize}
     	\item \textbf{Interpersonal meanings} are made by the \textbf{mood system}: \emph{imperative, interrogative, declarative, modality}
     	\item \textbf{Ideational meanings} are made by the \textbf{transitivity system}: \emph{predicates and their arguments}
     	\item \textbf{Textual meanings} are made by the \textbf{thematic system}: \emph{linking texts together with \emph{and}, \emph{but}, \emph{so} ... }
     \end{itemize}
     \end{frame}
 
  \begin{frame}
    \centering
    \includegraphics[width=0.60\textwidth]{../images/system.png}
  \end{frame}

\begin{frame}
\frametitle{Power and mood choice}
\begin{itemize}
	\item When there's a power disparity, some interpersonal meanings are less available to the one with less power.
	\item A good example is the mood system (Questions, statements, commands):
\begin{itemize}
	\item Me: Would you tell me the answer? / I think you should tell me the answer. / Tell me the readings!
	\item You: Can you speak slower? / We're having trouble understanding you / \underline{~~~~~~~~~~~} ?
\end{itemize}
\end{itemize}
\end{frame}


\begin{frame}
\frametitle{SFL and multimodality}
\begin{itemize}
\item Multimodality: the (simultaneous) use of different channels/modes to communicate
\item Inherent to online communication (page layout, at the very least)
\item We can look at images as also accomplishing the three metafunctions...
\end{itemize}
\end{frame}
 
  \begin{frame}
    \centering
    \includegraphics[width=0.50\textwidth]{../images/ikea.jpg}
  \end{frame}


\begin{frame}
\frametitle{SFL and genre}
\begin{itemize}
	\item In SFL, a genre is a staged, goal-oriented social process
	\item Culturally recognised sets of role relationships, things being spoken about, and modes of communication
	\item We can actually figure this stuff out from looking at the lexicogrammar
	\item Example: 
	\end{itemize}
	~~~~~~~~~~~~~~\emph{Submissions must contain 8--10 references.}
\end{frame}

  \begin{frame}
    \centering
    \includegraphics[width=0.50\textwidth]{../images/ikea.jpg}
  \end{frame}

\begin{frame}
	\frametitle{IPython Notebook ...}
	
\end{frame}


\begin{frame}
	\frametitle{Summary of findings}
	
	\begin{itemize}
	\item More imperatives
	\item More \emph{I would}
	\item More jargon
	\item More `metadiscourse'
	\item etc.
	\end{itemize}
\end{frame}


\begin{frame}
	\frametitle{What's causing the change?}
	
	Two contradictory hypotheses:

	\begin{enumerate}
	\item \textbf{Socialisation:} new members learn from veterans and implement the linguistic repertoire they've learned
	\item \textbf{Genre awareness:} new members know they are newcomers, and act like it
	\end{enumerate}
	... and what role does the technology itself play? 
\end{frame}


\begin{frame}
	\frametitle{Reasons for jargonisation?}

	We can use SFL's idea of metafunctions as an explanation for why a certain kind of language change occurs.
	
	\begin{itemize}
	\item \textbf{Interpersonal change:} jargon demonstrates expertise, belonging
	\item \textbf{Experiential change:} jargon distinguishes between hypertechnical things
	\item \textbf{Textual change:} shorter words are easier to type
	\end{itemize}

	The important thing to remember is that different kinds of meaning are made \emph{simultaneously.}
\end{frame}


\begin{frame}
	\frametitle{Brainstorming ...}

	What \emph{interpersonal}, \emph{experiential} and \emph{textual} meanings can we make by:
	
	\begin{itemize}
	\item Using imperatives: \emph{do this}, \emph{go there}
	\item Vague language: \emph{things will get better}, \emph{things are alright}
	\end{itemize}
\end{frame}


\begin{frame}
	\frametitle{Where to next?}
	\begin{itemize}
	\item More awareness of multimodality, hardware influence, site architecture
	\item Can we predict the influence of the three metafunctions?
	\item Can we \emph{measure} the influence?
	\item How does it play out in new technology? \emph{Reddit? iPhones?}
	\item A framework for doing SFL with mediated communication?
	\end{itemize}
\end{frame}


\begin{frame}
	\frametitle{Help!}
	
	~~~~~~~~~~~Thoughts?

\end{frame}







\begin{frame}[t,allowframebreaks]
\frametitle{References}
\bibliographystyle{apacite}
\bibliography{../references/libwin.bib}
\end{frame}

\end{document}
